\section{Implementation of the Backends}
Opie Backends have implementations for XML, VCF and SQL. We can
group the VCF and XML Backend in the File Family.

\section{Locking, Journal and Visibility}
Opie PIM encourages that all changes are made visible 
immediately and are propagated by Qt Signals by the Frontend
but we recognize that not every Backend
- specially the File Family - is capable of making every 
change physically available instantly or it would be too
complex to develop this is why this feature is nice to have
but not necessary implement by the Backend.\\
By default changes may not be lost by concurrent access and two different
saves. The feature to implement is called Auto-Update the Backends
can utilize QCOP for the communication. When one Backend saves the data
it needs to lock the access (it can use API), check if external changes
occurred and reload the data if necessary and then save the merged changes
and finally to give up the lock.
The user may call reload() any time and the access needs to be locked
for that. The Backends can use the appropriate techniques to provide
fast and secure operation.\\
The reload feature may be turned off for the Frontend so that a save()
might possible override changes that occurred.

\subsubsection{Journal and Out Of Memory}
The Backend need to make sure that it is capable of surviving crashes
and storage out of memory situation without losing data and changes.\\
The File Family is encouraged to use the journal on a per user basis 
to keep track of changes. The journal can be applied in case of a crash.\\
In out of disk storage situation the user needs to be informed so it might
free space.

\subsubsection{Features of the Backends}

\begin{center}
\begin{tabular}{|l|l|l|l|l|}
\hline
Backend & Live Update & Auto Update & Journal  & Available\\ \hline
XML & no  & yes & no but SQL & after save \\ \hline
SQL & yes & yes & yes        & immediately \\ \hline
VCF & no  & no  & no         & never      \\ \hline
\end{tabular}
\label{Feature of Backends}
\end{center}

\subsection{Generic Implementation}
\subsubsection{Recurrence}
FIXME
\subsubsection{OPimState}
FIXME
\subsubsection{Reminders and Alarms}

\subsection{Todo}
\subsubsection{XML}

% supported attributes
\begin{tabular}{|l|l|}
\hline
Key            & Possible Values \\ \hline
Categories     & Comma separated list of Integers \\ \hline
Uid            & Unique negative Integer generated by timet \\ \hline
HasDate        & 0 for false and 1 for true \\ \hline
Completed      & 1 for completed and 0 for not completed \\ \hline
Description    & Escaped multi line Text \\ \hline
Summary        & One line summary \\ \hline
Priority       & Between 0 and 5, 5 high priority \\ \hline
DateDay        & The Day for the DueDate \\ \hline
DateMonth      & The Month for the DueDate \\ \hline
DateYear       & The year for the DueDate \\ \hline
Progress       & 0,10,20,30,40,50,60,70,80,90,100 \\ \hline
CompleteDate   & YYYYMMDD formated Date \\ \hline
StartDate      & YYYYMMDD formated Date \\ \hline
CrossReference & TODO!!  \\ \hline
State          & Relates to OPimState 0-4 arre current values \\ \hline
Alarms         & YYYYMMDD:durationOfTheAlarm:sound:future 
                 multiple alarams are joined with ; \\ \hline
Reminders      & Notused TODO!!! \\ \hline
rtype          & Daily, Weekly, MonthlyDay, MonthlyDate, Yearly, None\\ \hline
rweekdays      & Char OPimReccruence::Days ored together \\ \hline
rposition      & The day in the week TODO\\ \hline
rfreq          & Every n (as int) \\ \hline
start          & The creation of the recurrence in UTC time\\ \hline
rhasenddate    & 1 if the recurrence has an end \\ \hline
enddt          & The end date of the recurrence from UTC time\\ \hline
\end{tabular}

\subsubsection{VCF}


\subsubsection{SQL}


\subsection{Contact}
\subsubsection{XML}
\subsubsection{VCF}
\subsubsection{SQL}


\subsection{Datebook}
\subsubsection{XML}
\subsubsection{VCF}
\subsubsection{SQL}

